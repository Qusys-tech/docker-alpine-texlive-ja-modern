\documentclass[a4paper,xelatex,ja=standard]{bxjsarticle}
\usepackage{metalogo} %各種logo
\usepackage{amsmath,amssymb}
\usepackage{here}
\usepackage{graphicx}
\usepackage{url}
\usepackage{hyperref}

\usepackage{physics}
\usepackage{algorithm, algpseudocode}
\usepackage{bm}

\usepackage{tikz}
\usetikzlibrary{positioning}
% tikzマクロ 
% https://www.opt.mist.i.u-tokyo.ac.jp/~tasuku/tikz.html を参考にした.
% https://eqseqs.hatenablog.com/entry/2019/10/11/092852 も見るとよい
\newcommand{\highlight}[2][yellow]{\tikz[baseline=(x.base)]{\node[rectangle,rounded corners,fill=#1!10](x){$\displaystyle #2$};}}
\newcommand{\highlightcap}[3][yellow]{\tikz[baseline=(x.base)]{\node[rectangle,rounded corners,fill=#1!10](x){$\displaystyle #2$} node[below of=x, color=#1]{#3};}}

\begin{document}
    \title{{\XeLaTeX} sample}
    \author{Qusys}
    \maketitle

    This is a sample of {\XeLaTeX}.
    これは{\XeLaTeX}のサンプル.

    \section{Physics package test}
    一般に流体は圧縮性を持つが, 圧縮性が無視できる遅い流れは単純化のために,
    Physics packageを用いるとベクトルや偏微分が簡単にタイプセットできる.
    Navier Stokes方程式\eqref{eq:ns}と連続の式\eqref{eq:continuous}を例に示す.
    \begin{align}
        \pdv{\vb*{u}}{t} +(\vb*{u}\cdot \grad)\vb*{u} &= -\frac{1}{\rho}\grad p +\nu \laplacian\vb*{u}
        \label{eq:ns}\\
        \div \vb*{u} &= 0
        \label{eq:continuous}
    \end{align}

    \section{Tikz package Test}
    最小二乗法は当てはめたい関数を
    \begin{align}
        f(\vb*{x}_i) = \sum_{j=1}^n a_{ij}x_{ij}
    \end{align}
    とすると係数の$a_{ij}$を求める. ベクトル表記すると
    \begin{align}
        \highlightcap[red]{\vb*{a}}{未知} \cdot \highlightcap[blue]{\vb*{x}_i}{既知}
         = \highlightcap[blue]{y_i}{既知}
    \end{align}
    となる.
\end{document}